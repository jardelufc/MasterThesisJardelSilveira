\chapter*{Abstract}\addcontentsline{toc}{chapter}{Abstract}
\markboth{}{}

\PARstartOne{R}{eal-time} embedded systems have severe dependability restrictions such as: time, consumption, weight and total taken volume requirements. Considering the design of integrated circuits (ICs), in order to reach the reliability level required by these systems, one may use specific integrated circuits manufacturing processes, such as SOI/SOS technology (Silicon on Insulator/Silicon on Sapphire) that is inherently resistant to radiation and primarily used in military and aerospace applications. Another approach involves applying dependability techniques in the integrated circuit design phase. Furthermore, at software and system level (i.e. in the integration of electronic components in a printed circuit board) it is also possible to apply techniques to increase the reliability of these systems.
	The integrated circuits manufactured through specific processes have high cost due to their small production volume. Besides those specific-process ICs are in general three generations late in relation to the state of the art. Therefore, it is preferable to make the design of a fault tolerant integrated circuit and be able to manufacture it by a state of the art process.
	On the other hand, from the point of view of software development, the resources provided by the programming language used to program such systems may influence the system reliability level. The use of high abstraction level languages like Java, may diminish the occurrence of programming errors and decrease the number of non-predicted faults inserted during the development phase. The  JOP (Java Optimized Processor) soft ip core for FPGAs (Field Programmable Gate Array) is an optimized implementation of the Java virtual machine, in the scope of hardware, for real-time applications.
	In this work we propose hardware design techniques for JOP soft ip core, which detect and correct errors in SRAM (Static Random Access Memory) memory code area. The proposed techniques increase system reliability and leave the features of real-time of JOP soft ip core untouched.
    The modified soft ip core JOP was implemented on a Virtex 4 FPGA and both the operating frequency and number of logic gates are compared to the original soft ip core JOP.
    
\textbf{Keywords:} fault tolerance, Java processor, real-time embedded systems, CMOS, integrated circuit.


